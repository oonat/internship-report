I started my internship without any knowledge about recommender systems. Throughout the internship, I experienced first hand the major challenges such as sparsity, scalability, cold start, etc. that recommender systems face and observed the mainstream approaches developed to overcome these issues. Moreover, I analysed the state-of-the-art algorithms designed for recommendation systems with their strengths and weaknesses.

As the next step, we needed to test the performances of the implemented recommenders. During the testing phase, I familiarized myself with different cross-validation types such as Leave-one-out, K-Fold, etc. and applied these methods to determine the performance of recommenders I built. Furthermore, I acquired profound knowledge about popular evaluation metrics and the situations in which they are preferred. For example, RMSE and MAE are widely-used metrics for datasets containing explicit ratings while hit rate and map@k which is especially useful if we care about the order of recommended products are preferred for datasets with implicit ratings. 

From the technical aspect, as a result of working with Neo4j, I learnt the basics, pros, and cons of graph databases. Especially the data visualization feature of Neo4j was really helpful to analyse the dataset. The first versions of the recommenders I implemented were running very slowly due to lots of unnecessary iterations, the solution was performing operations with a matrix approach using Numpy. This experience showed me the efficiency of Numpy coming from being written based on C. Since one of the recommenders is graph-based, I got to know python libraries with graph utilities such as Scipy and Scikit-learn which provided information that could be also useful in future projects.

On the whole, the internship has provided me new insights into the recommender systems and in general data science. Not only the internship developed my technical skills and perspective but also it allowed me to see the internal dynamics of the technologies around me.  For instance, now I have an idea of how the shopping platform I visit makes recommendations or why sites like Linkedin and Twitter ask new members to follow someone as soon as they register. \\