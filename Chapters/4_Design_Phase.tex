\subsubsection{For Eigentrust Weighted Recommender}
As preliminary information, the Neo4j database on which we tested the recommender, initially contained customer, product, and transaction records. After using "community detection" and "Eigentrust" modules provided by TACoRec, disjoint customer communities and Eigentrust scores between the customers belonging to the same community were also added to the database.\\

Eigentrust represents how strongly connected the customers are to their communities (for the detailed information, please check section \ref{about_eigentrust}) and stored as a property of the relationship between the customer and his/her community in the Neo4j database. We can draw two conclusions from this information;
\begin{enumerate}
	\item Since the Eigentrust scores are only calculated between the members of a community while making a recommendation to a customer, only the preferences of people in the same community as that customer should be taken into account.
	\item Although the opinions of stereotype customers (i.e. customers with higher Eigentrust score) are important, we should also consider the similarity between the target user and the users who make suggestions to keep recommendations personalized.
\end{enumerate}
Based on these conclusions, I designed two modules: Recommender and Filterer. The Recommender is responsible for writing the recommendations generated by the Filterer module to the database while the remaining weight of the project is carried by the Filterer: getting transaction and Eigentrust records from the database via Neo4j driver, measuring similarities between customers based on transaction records, calculating recommendation coefficients for each product and generating recommendations according to these coefficients. \\

For the detailed information and the recommender structure, please check section \ref{eigentrust_section} and Figure \ref{fig:eigentrust_structure}.
\subsubsection{For Inverse Distance Weighted Recommender}
"Inverse Distance Weighting" method is inspired by an article\cite{inverse_article} that suggests calculating the trust scores between customer pairs by the reciprocal of the shortest distance between them on a trust network. Since I worked on datasets with no explicit trust information, I decided to use transaction records to build the trust network. \\

So with this approach, In addition to Filterer and Recommender, I implemented the Graph Module which is responsible for building the graph according to the preferred method (unweighted or euclidean distance weighted) from the transaction records and calculating trust scores by taking the reciprocal of the shortest distances between customers in the graph. \\

For the detailed information and the recommender structure, please check section \ref{inverse_section} and Figure \ref{fig:inverse_structure}.