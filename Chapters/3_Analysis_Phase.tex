The variety and number of products are increasing day by day, which creates the problem of recommending the most appropriate products for users. One of the main approaches used in the design of recommender systems is "collaborative filtering". The approach uses prior behaviours of customers such as rating profiles, product preferences, etc. to generate recommendations. Collaborative filtering methods can be classified according to which factor they prioritize while making suggestions. In this project, we focused on "trust-based collaborative filtering" which generates recommendations considering the trust between users. 

The definition and calculation of trust may differ in many sources and researches. For instance, \cite{Eigentrust} approaches the issue from the probabilistic aspect and calculates trust through successful and unsuccessful transactions, the trust metric used in \cite{papagelis_article} is based on "Pearson Correlation Similarity" between users while \cite{lathia_article} stress the value of providing ratings and argued that users who give more rates are more trustworthy, even if they don't rate similarly. Such recommendation systems aim to calculate trust scores from behaviours of customers (e.g. ratings) to make good recommendations in the absence of a trust network.

On the other hand, there are also approaches\cite{massa_article} that are developed to operate on datasets containing explicit trust scores (e.g. Epinions \cite{Epinions}) and make suggestions by using these scores directly or by combining them with additional features such as similarity, product or user attributes (especially in hybrid recommenders), etc. 

During the internship, due to lack of explicit trust information in most systems, we mainly focused on developing accurate trust-based recommender systems working on datasets with no explicit trust information such as Stockmount, Amazon Food Review, etc. Some of these datasets (e.g. Stockmount) contain implicit ratings while the rest of them have explicit ratings given by customers.

For the "Eigentrust Weighted Recommender", we were able to use "community detection" and "Eigentrust calculation" modules provided by TACoRec. The part I worked on was integrating the "Eigentrust scores" to the collaborative filtering algorithm. 

For the "Inverse Distance Weighted Recommender", I experienced the difficulty of inferring trust from implicit customer behaviours which was I believe the most meaningful and challenging part of the internship.
