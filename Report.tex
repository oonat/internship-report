\documentclass[12pt]{article}
\usepackage[english]{babel}
\usepackage[utf8x]{inputenc}
\usepackage{amsmath}
\usepackage{graphicx}
\graphicspath{{Images/}}
\usepackage[colorinlistoftodos]{todonotes}
\usepackage{listings}
\usepackage{hyperref}
\graphicspath{ {images/} }

\usepackage{xcolor}

\definecolor{codegreen}{rgb}{0,0.6,0}
\definecolor{codegray}{rgb}{0.5,0.5,0.5}
\definecolor{codepurple}{rgb}{0.58,0,0.82}
\definecolor{backcolour}{rgb}{0.95,0.95,0.92}

\lstdefinestyle{mystyle}{
	backgroundcolor=\color{backcolour},   
	commentstyle=\color{codegreen},
	keywordstyle=\color{magenta},
	numberstyle=\tiny\color{codegray},
	stringstyle=\color{codepurple},
	basicstyle=\ttfamily\footnotesize,
	breakatwhitespace=false,         
	breaklines=true,                 
	captionpos=b,                    
	keepspaces=true,                 
	numbers=left,                    
	numbersep=5pt,                  
	showspaces=false,                
	showstringspaces=false,
	showtabs=false,                  
	tabsize=2
}

\lstset{style=mystyle}

\begin{document}
\begin{titlepage}
\newcommand{\HRule}{\rule{\linewidth}{0.1mm}} 
\center 
\includegraphics[scale=0.12]{ceng.jpg} \\
\vspace{1cm}
\textsc{\Large MIDDLE EAST TECHNICAL UNIVERSITY}\\[0.5cm]
\textsc{\large Department Of Computer Engineering}\\[0.5cm]
\textsc{\large CENG 300}\\
\HRule \\[0.4cm]
{ \huge \Large Summer Practice Report }\\[0.4cm]
METU Data Mining Research Group \\
Start Date: \hspace{25px}  End Date: \\
Total Working Dates:

\HRule \\[0.3cm]
{\today}\\[1.2cm]


\begin{minipage}{0.4\textwidth}
\begin{flushleft} \large

\emph{Student:}\\
Onat ÖZDEMİR
\end{flushleft}


\end{minipage}
\begin{minipage}{0.4\textwidth}
\begin{flushright} \large
\emph{Instructors:} \\
Prof.Dr.Pınar KARAGÖZ \\
\end{flushright}
\begin{flushright} \large
Prof.Dr.İsmail Hakkı TOROSLU \\
\end{flushright}
\end{minipage}\\[1cm]
\vspace{2cm}


Student's Signature \hspace{150px} Organization Approval \\


\end{titlepage}

\include{GradingRubric}
\tableofcontents          
\newpage

\section{Introduction}
I have done my summer internship at METU Data Mining Research Group under the supervision of Prof.Dr.Pınar KARAGÖZ and Prof.Dr.İsmail Hakkı TOROSLU. The task I have worked on was implementing a Trust Based Recommender using the collaborative filtering method and testing it on provided dataset. The dataset contains information about customers and the products they have bought. In addition to dataset, I was able to use eigentrust calculation and community detection modules provided by the TACOREC.

\section{Project}
During the internship, I implemented two different trust based recommenders
\begin{enumerate}
\item Eigentrust Weighted Recommender
\item Inverse Distance Weighted Recommender
\end{enumerate} 
The details of these two recommenders can be found in section 2.4 and 2.5, respectively.

\subsection{Analysis Phase}
There were two problems I need to solve: \\
1 - Dataset was very sparse \\ 
2 - There was no explicit trust information \\
In the both implementations I have made, first case handled by filtering the customers and products which purchased and were bought more than filtering threshold times. To gain better understanding on the second problem, I studied implicit trust calculation methods and looked into lots of research papers.

\subsection{Design Phase}
\subsection{Implementation Phase}
Since there are two different implementations, I have divided the details of the recommenders into two subsections: section 2.4 and 2.5. Under this subsection, libraries and technologies used in implementations are explained.
\subsubsection{Neo4j}
\paragraph{Driver Installation}:
\begin{lstlisting}[language=bash]
pip install neo4j
\end{lstlisting}

\paragraph{Configuration}:
\begin{lstlisting}[language=python]
import neo4j

...
uri = self._config["database"]["neo4j"]["uri"]
user = self._config["database"]["neo4j"]["user"]
password = self._config["database"]["neo4j"]["password"]

self._driver = neo4j.Driver(uri, auth=(user, password))

\end{lstlisting}

\paragraph{Sample Usage}:
\begin{lstlisting}[language=python, caption=Neo4j cypher example]
import neo4j
...
    def get_customer_trust(self, customer_id):

		query = (
			f"MATCH (u:Customer)-[r:BELONGS_IN]->(:Community) "
			f"WHERE u.id = {repr(customer_id)} "
			f"RETURN r.eigentrust"
		)

		with self._driver.session() as session:
			return tuple(session.run(query).single())

\end{lstlisting}

\subsubsection{Numpy}
\paragraph{Installation}:
\begin{lstlisting}[language=bash]
pip install numpy
\end{lstlisting}

\paragraph{Sample Usage}:
\begin{lstlisting}[language=python, caption=Numpy example]
import numpy as np

class TrustBasedFilterer(object):
...

	def _create_customers_versus_products_table(self):

		self._customers_versus_products_table = np.zeros(
				(self._unique_customers.shape[0],
				self._unique_products.shape[0]),
				dtype=np.bool,
			)
			
		self._customers_versus_products_table[
				self._sales[:, 0],
				self._sales[:, 1],
			] = True
\end{lstlisting}

\subsubsection{Scipy}
\paragraph{Installation}:
\begin{lstlisting}[language=bash]
pip install scipy
\end{lstlisting}

\paragraph{Sample Usage}:
\begin{lstlisting}[language=python, caption=Scipy example]
from scipy.sparse import csr_matrix
from scipy.sparse.csgraph import dijkstra

class Graph(object):
...

	def _create_distance_matrix(self):

		self._create_adjacency_matrix()

		self._adjacency_matrix = \
			csr_matrix(self._adjacency_matrix)
		
		self._distance_matrix = dijkstra( 
			csgraph=self._adjacency_matrix, 
			directed=False, 
			return_predecessors=False, 
			unweighted=True,
			limit=self._max_distance)

		self._distance_matrix\ 
			[~np.isfinite(self._distance_matrix)] = 0
\end{lstlisting}


\subsection{Eigentrust Weighted Trust Based Recommender}
\subsection{Inverse Distance Weighted Trust Based Recommender}
Inverse Distance Weighted Trust Based Recommender consists of three different modules: \\
\begin{enumerate}
\item \textbf{graph.py}: Responsible for creating adjacency matrix from given customer versus product matrix, calculating the distances between customers applying djkstra algorithm to the adjacency matrix using scipy library, and calculating the trust by taking the recreciprocal of the distance matrix.
\item \textbf{trust\_based\_filterer.py}: The module initially takes list of unsorted and unfiltered transaction list and filters it to create more denser customer versus product matrix. Then creates a graph object by giving the customer versus product matrix as parameter and use the trust matrix created by the graph object to calculate recommendation coefficient for each product. Finally, for each user, the module sorts all products with respect to their recommendation coefficients in descending order and recommends top k of them.
\item \textbf{trust\_based\_recommender.py}: This module sends the list of transactions to the trust\_based\_filterer and takes the list of recommendations for each customer in the database as a response. The main task of this module is writing the recommendations to the database using neo4j driver.
\end{enumerate}
\subsection{Testing Phase}
\subsubsection{Surprise}
\paragraph{Installation}:
\begin{lstlisting}[language=bash]
 pip install scikit-surprise
\end{lstlisting}

\paragraph{Sample Usage}:
\begin{lstlisting}[language=python, caption=Surprise example]
from surprise import AlgoBase, PredictionImpossible, Dataset
from surprise.model_selection import cross_validate

class Inverse_distance_weighted_tbr(AlgoBase):
...

reader = reader = Reader(line_format='user item rating timestamp', sep=';', rating_scale=(1, 5))

data = Dataset.load_from_file('./dataset.csv', reader=reader)
algo = Inverse_distance_weighted_tbr()

cross_validate(algo, data, cv=5, verbose=True)
\end{lstlisting}

\subsubsection{Matplotlib}
\paragraph{Installation}:
\begin{lstlisting}[language=bash]
pip install matplotlib
\end{lstlisting}

\paragraph{Sample Usage}:
\begin{lstlisting}[language=python, caption=Matplotlib example]
import matplotlib.pyplot as plt
...
plt.hist(eigentrust_list, 
		 color = 'blue', 
		 edgecolor = 'black',
		 bins = 1000)
plt.title('Distribution of the Eigentrust Values')
plt.xlabel('Range')
plt.ylabel('Number of Eigentrust Values in the range')
plt.show() 
\end{lstlisting}

\subsubsection{Leave-one-out Cross Validation}

\section{Organization}
\subsection{METU Data Mining Research Group}

\section{Conclusion}

\bibliographystyle{plain}
\bibliography{Bibliography.bib}


\end{document}